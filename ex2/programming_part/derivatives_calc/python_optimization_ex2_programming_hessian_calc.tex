\documentclass[fleqn]{article}
\usepackage{amsmath}
\usepackage{amsfonts} 
\usepackage{graphicx}
\usepackage{tabularx}

\begin{document}
\newcommand\tab[1][1cm]{\hspace*{#1}}

\underline{In general}: \\

\[
\nabla^2 f(x) =
\begin{bmatrix} 
	\frac{\partial^2 f}{\partial x_1^2} & \frac{\partial^2 f}{\partial x_1 \partial x_2} & \cdots & \frac{\partial^2 f}{\partial x_1 \partial x_n}\
\\ 
	\frac{\partial^2 f}{\partial x_1 \partial x_2} & \frac{\partial^2 f}{\partial x_2^2} & \cdots & \frac{\partial^2 f}{\partial x_2 \partial x_n} \\ 
	\vdots & \vdots & \ddots & \vdots \\ 
	\frac{\partial^2 f}{\partial x_1 \partial x_n} & \frac{\partial^2 f}{\partial x_2 \partial x_n} & \cdots & \frac{\partial^2 f}{\partial x_n^2} \\ 
\end{bmatrix}
\]\\ \\

\underline{\textbf{Function b.iii}}:\\
\begin{multline*}
Q = 
\begin{bmatrix} 
	\frac{\sqrt{3}}{2} & \frac{1}{2} \\ 
	-\frac{1}{2} & \frac{\sqrt{3}}{2} \\ 
\end{bmatrix}
\begin{bmatrix} 5 & 0 \\ 0 & 1 \\ \end{bmatrix}
\begin{bmatrix} 
	\frac{\sqrt{3}}{2} & -\frac{1}{2} \\ 
	\frac{1}{2} & \frac{\sqrt{3}}{2} \\ 
\end{bmatrix}
=
\begin{bmatrix} 
	\frac{5\sqrt{3}}{2} & \frac{1}{2} \\ 
	-\frac{5}{2} & \frac{\sqrt{3}}{2} \\ 
\end{bmatrix}
\begin{bmatrix} 
	\frac{\sqrt{3}}{2} & -\frac{1}{2} \\ 
	\frac{1}{2} & \frac{\sqrt{3}}{2} \\ 
\end{bmatrix}
=
\begin{bmatrix} 
	\frac{5\sqrt{3}}{2} \cdot \frac{\sqrt{3}}{2} + \frac{1}{4} &
	-\frac{5 \sqrt{3}}{4} + \frac{\sqrt{3}}{4} \\ 
	-\frac{5 \sqrt{3}}{4} + \frac{\sqrt{3}}{4}                 &
	\frac{5}{4} + \frac{3}{4} \\ 
\end{bmatrix}
= \\
\begin{bmatrix} 
	4         &	-\sqrt{3} \\ 
	-\sqrt{3} &2          \\ 
\end{bmatrix}
\end{multline*}

\begin{multline*}
f(x) = x^T Q x =
\begin{pmatrix} x_1 & x_2 \\ \end{pmatrix}
\begin{pmatrix} 
	4         &	-\sqrt{3} \\ 
	-\sqrt{3} &2          \\ 
\end{pmatrix}
\begin{pmatrix} x_1 \\ x_2 \end{pmatrix}
=
\begin{pmatrix} 4x_1 -\sqrt{3}x_2 & -\sqrt{3}x_1+2x_2 \\ \end{pmatrix}
\begin{pmatrix} x_1 \\ x_2 \end{pmatrix}
= \\
4x_1^2 -\sqrt{3} x_1 x_2 - \sqrt{3} x_1 x_2 +2x_2^2 
=
4x_1^2 - 2\sqrt{3} x_1 x_2 +2x_2^2 
\end{multline*}

\begin{multline*}
\nabla f(x) =
\begin{bmatrix}
	8x_1 -2\sqrt{3}x_2 \\ 4x_2-2\sqrt{3}x_1
\end{bmatrix} \\
\nabla^2 f(x) =
\begin{bmatrix} 
	8 & -2\sqrt{3} \\ -2\sqrt{3} & 4 \\ 
\end{bmatrix} \\
\end{multline*} \\

\underline{\textbf{Function c.Rosenbrock}}:\\
\begin{multline*}
f(x) = 100(x_2-x_1^2)^2+(1-x_1)^2 = \\
100x_2^2-200x_1^2x_2+100x_1^4+1-2x_1+x_1^2 = \\
100x_1^4+x_1^2-2x_1-200x_1^2x_2+100x_2^2\\
\end{multline*}

\begin{multline*}
\nabla f(x) =
\begin{bmatrix}
	400x_1^3+2x_1-2-400x_1x_2 \\ 200x_2-200x_1^2
\end{bmatrix} \\
\nabla^2 f(x) =
\begin{bmatrix} 
	1200x_1^2+2-400x_2 & -400x_1 \\
	-400x_1            & 200     \\ 
\end{bmatrix} \\
\end{multline*} \\

\underline{\textbf{Function qp (ex.2)}}:\\
$f_0(x,y,z) = x^2+y^2+(z+1)^2$ \\
$h_1(x,y,z) = x+y+z-1 \; (=0)$ \\
$f_1(x,y,z) = -x $ \\
$f_2(x,y,z) = -y $ \\
$f_3(x,y,z) = -z $ \\

\underline{Function for newton method}:\\

From lecture 7+8, slide 61: \\
Let $f^{new}(x,y,z) = tf_0(x,y,z)+\phi(x,y,z)$\\

$f^{new}(x,y,z) = t(x^2+y^2+(z+1)^2)-log(x)-log(y)-log(z) = $\\ 

$t(x^2+y^2+z^2+2z+1)-log(x)-log(y)-log(z)$\\

\begin{multline*}
\nabla f^{new}(x,y,z) =
\begin{bmatrix}
	2tx-\frac{1}{x} \\
	2ty-\frac{1}{y} \\
	2tz+2t-\frac{1}{z}
\end{bmatrix} \\
\nabla^2 f^{new}(x,y,z) =
\begin{bmatrix} 
	2t+\frac{1}{x^2} & 0                & 0                \\
	0                & 2t+\frac{1}{y^2} & 0                \\
	0                & 0                & 2t+\frac{1}{z^2} \\ 
\end{bmatrix} \\
\end{multline*} \\

Let $\phi_i(x)=-log(-f_i(x))$ \\

$\phi_1(x,y,z) = -log(x)$
\begin{multline*}
\nabla \phi_1(x,y,z) =
\begin{bmatrix} -\frac{1}{x} \\ 0 \\ 0 \end{bmatrix} \\
\end{multline*} \\
$\nabla^2 \phi_1(x,y,z) = diag(\frac{1}{x^2},0,0)$ \\


$\phi_2(x,y,z) = -log(y)$ \\
\begin{multline*}
\nabla \phi_2(x,y,z) =
\begin{bmatrix} 0 \\ -\frac{1}{y} \\ 0 \end{bmatrix} \\
\end{multline*}
$\nabla^2 \phi_2(x,y,z) = diag(0,\frac{1}{y^2},0)$ \\


$\phi_3(x,y,z) = -log(z)$ \\
\begin{multline*}
\nabla \phi_3(x,y,z) =
\begin{bmatrix} 0 \\ 0 \\ -\frac{1}{z} \end{bmatrix} \\
\end{multline*} \\
$\nabla^2 \phi_3(x,y,z) = diag(0,0,\frac{1}{z^2})$ \\



\underline{\textbf{Function lp (ex.2)}}:\\
$f_0(x,y) = -x-y$ \\
$f_1(x,y) = -x-y+1 $ \\
$f_2(x,y) = y-1 $ \\
$f_3(x,y) = x-2 $ \\
$f_4(x,y) = -y $ \\

\underline{Function for newton method}:\\

From lecture 7+8, slide 61: \\
Let $f^{new}(x,y) = tf_0(x,y)+\phi(x,y)$\\

$f^{new}(x,y) = -tx-ty-log(x+y-1)-log(1-y)-log(2-x)-log(y)$\\ 

\begin{multline*}
\nabla f^{new}(x,y) =
\begin{bmatrix}
	-t -\frac{1}{x+y-1}+\frac{1}{2-x}             \\
	-t -\frac{1}{x+y-1}+\frac{1}{1-y} -\frac{1}{y}
\end{bmatrix} \\
\nabla^2 f^{new}(x,y) =
\begin{bmatrix} 
    \frac{1}{(x+y-1)^2} + \frac{1}{(2-x)^2} &
    \frac{1}{(x+y-1)^2}
 \\
	\frac{1}{(x+y-1)^2}                                 &
	\frac{1}{(x+y-1)^2}+\frac{1}{(1-y)^2}+\frac{1}{y^2}
 \\
\end{bmatrix} \\
\end{multline*} \\

Let $\phi_i(x)=-log(-f_i(x))$ \\

$\phi_1(x,y) = -log(x+y-1) $ \\
\begin{multline*}
\nabla \phi_1(x,y) =
\begin{bmatrix}
	-\frac{1}{x+y-1}  \\ -\frac{1}{x+y-1}
\end{bmatrix} \\
\nabla^2 \phi_1(x,y) =
\begin{bmatrix} 
    \frac{1}{(x+y-1)^2} & \frac{1}{(x+y-1)^2}
 \\
	\frac{1}{(x+y-1)^2} & \frac{1}{(x+y-1)^2}
 \\
\end{bmatrix} \\
\end{multline*} \\

$\phi_2(x,y) = -log(1-y) $ \\
\begin{multline*}
\nabla \phi_2(x,y) =
\begin{bmatrix}	0   \\ \frac{1}{1-y} \end{bmatrix} \\
\end{multline*} \\
$\nabla^2 \phi_2(x,y) =diag(0, \frac{1}{(1-y)^2} )$ \\

$\phi_3(x,y) = -log(2-x) $ \\
\begin{multline*}
\nabla \phi_3(x,y) =
\begin{bmatrix}	\frac{1}{2-x}   \\ 0 \end{bmatrix} \\
\end{multline*} \\
$\nabla^2 \phi_3(x,y) =diag(\frac{1}{(2-x)^2}, 0)$ \\

$\phi_4(x,y) = -log(y) $ \\
\begin{multline*}
\nabla \phi_4(x,y) =
\begin{bmatrix}	0 \\ -\frac{1}{y} \end{bmatrix} \\
\end{multline*} \\
$\nabla^2 \phi_4(x,y) =diag(0, \frac{1}{y^2})$ \\

\end{document}\\