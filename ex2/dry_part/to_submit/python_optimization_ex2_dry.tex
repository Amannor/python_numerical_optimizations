\documentclass[fleqn]{article}
\usepackage{amsmath}
\usepackage{amsfonts} 
\usepackage{graphicx}
\usepackage{tabularx}
\usepackage{listings}

\begin{document}
\newcommand\tab[1][1cm]{\hspace*{#1}}
\section*{Numerical Optimization with Python -  Ex. 2: dry part}
Alon\\ \\

The code I used to create the drawings can be found at:\\

\begin{lstlisting}[breaklines]
https://github.com/Amannor/python_numerical_optimizations/tree/main/ex2/dry_part
\end{lstlisting}

\underline{\textbf{Question 1}}:\\

\textbf{1.1} \\
See figure below. \\

\begin{figure}[h!]
\includegraphics[width=0.8\linewidth]{q1_1.PNG}
\caption{Q1.1: contour lines of $\lvert x\rvert +\lvert y \rvert$}
\end{figure}

\textbf{1.2} \\
I wasn't able to draw everything in a single figure, so see 2 figure below:\\
- The feasible region on top of the contour of the objective function. \\
- The contours of the constraints (without the objective function). \\

The constraint $(x-1)^2+(y-1)^2 \leq 1$ yields a circle of radius 1 with a center in (1,1). Adding to it the constraint $y \leq 1$ makes the said circle a (lower) half circle.\\

*I used the following link to find the minimum value: \\
\begin{lstlisting}[breaklines]
https://www.wolframalpha.com/input/?i=min+%7Cx%7C+%2B%7Cy%7C%2C+%28x-1%29%5E2+%2B+%28y-1%29%5E2%3C%3D1%2C+y%3C%3D1
\end{lstlisting} 

\begin{figure}[h!]
\includegraphics[width=0.8\linewidth]{q1_2.PNG}
\caption{Q1.2: Contour lines of the objective function $f_0(x,y) = \lvert x\rvert +\lvert y \rvert$ and the the constraints (feasible region is the black half circle)}
\end{figure}


\begin{figure}[h!]
\includegraphics[width=0.8\linewidth]{q1_2_constraints_contours.PNG}
\caption{Q1.2: Contour lines of the constraints functions}
\end{figure}

\textbf{1.3} \\
See figure below. \\

Since the minimum of the function  $\lvert x\rvert +\lvert y \rvert $ is at the origin (0,0) and it's monotonically increasing as the (absolute) values of x and y grow, then the minimum in the feasible region is achieved in the closest point to the origin. This point is $(1-\frac{1}{\sqrt{2}}, 1-\frac{1}{\sqrt{2}})$ and is marked in the figure below.\\

The value of the function at this point is $\lvert 1-\frac{1}{\sqrt{2}} \rvert + \lvert1-\frac{1}{\sqrt{2}}\rvert = 2 - \sqrt{2}$\\

\begin{figure}[h!]
\includegraphics[width=0.8\linewidth]{q1_3.PNG}
\caption{Q1.3: The minimum point in the feasible region is marked in blue}
\end{figure}

\textbf{1.4} \\
The constraints are: \\
$f_1 (x,y) = (x-1)^2+(y-1)^2-1 \leq 0$ \\
$f_2 (x,y) = y-1 \leq 0$ \\

We recall that by definition, at a feasible point (x,y), an inequality constraint $ i \in \mathcal{I}$ is said to be active if $f_i (x,y) = 0$ and inactive if $f_i (x,y) < 0$ (lecture 4, slide 7).\\
Hence we'll just assign the value of the minimum point we got above: \\
\begin{multline*}
f_1(1-\frac{1}{\sqrt{2}}, 1-\frac{1}{\sqrt{2}}) = 
(1-\frac{1}{\sqrt{2}}-1)^2 + (1-\frac{1}{\sqrt{2}}-1)^2 -1 = 
\frac{1}{2}+\frac{1}{2}-1 = 0 \\
\rightarrow \boxed{f_1 \; is \; active}
\end{multline*} \\

\begin{multline*}
f_2(1-\frac{1}{\sqrt{2}}, 1-\frac{1}{\sqrt{2}}) = 
1-\frac{1}{\sqrt{2}}-1 = -\frac{1}{\sqrt{2}} < 0 \\
\rightarrow \boxed{f_2 \; is \; not\; active}
\end{multline*} \\

\textbf{1.5} \\
See figure below. \\

\begin{figure}[h!]
\includegraphics[width=0.8\linewidth]{q1_5.PNG}
\caption{Q1.5: Gradients at minimum point: black is of the objective function and green is of the first constraint $f_1 = (x-1)^2+(y-1)^2-1 \leq 0$}
\end{figure}




\clearpage\underline{\textbf{Question 2}}:\\
\textbf{2.1} \\
The graph of the function $f_0(x,y) = 0.5x-y$ is a plane.

I wasn't able to draw everything in a single figure, so see 2 figure below:\\
- The feasible region on top of the contour of the objective function. \\
- The contours of the constraints (without the objective function). \\

Note that for some reason the bottom line of the polygon of the feasible region didn't come out parallel to the x axis as it was suppose to. I'm not sure as to why is that. I went over my code several times and I think it might be related to overflow of values that can't be exactly stored in binary form (e.g. $\frac{1}{3}$).\\

\begin{figure}[h!]
\includegraphics[width=0.8\linewidth]{q2_1.PNG}
\caption{Q2.1: Contours lines of the objective function $0.5x-y$ and the feasible region (in black) given by the constraints}
\end{figure}

\begin{figure}[h!]
\includegraphics[width=0.8\linewidth]{q2_1_constraints_contours.PNG}
\caption{Q2.1: Contours lines of the constraints functions}
\end{figure}


\textbf{2.2} \\
The minimum value is $x^* = (1,2)$ and the optimal value is $p^* = f_0(1,2) = 0.5 \cdot 1 -2 = -1.5$. \\
See figure below.\\

*I used the following link to find the minimum value: \\
\begin{lstlisting}[breaklines]
https://www.wolframalpha.com/input/?i=min+0.5x-y%2C+-x%2By-1%3C%3D0%2C+-x%2F3%2By-5%2F3%3C%3D0%2C+x-4%3C%3D0%2C+y-3%3C%3D0%2C+x%3E%3D0%2C+y%3E%3D0
\end{lstlisting} 

\begin{figure}[h!]
\includegraphics[width=0.8\linewidth]{q2_2.PNG}
\caption{Q2.2: minimum point (1,2) in feasible region is marked in blue}
\end{figure}

\textbf{2.3} \\
We'll go over each constraint function and calculate.\\
Of course I wrote (arranged the variables in) each function s.t. the constraint will be $f_i \leq 0 \; (i=1,...,6)$\\

\begin{multline*}
\underline{f_1(x,y) = -x+y-1} \\
f_1(1,2) = -1+2-1 = 0 \rightarrow \boxed{f_1 \; is \; active}
\end{multline*} \\

\begin{multline*}
\underline{f_2(x,y) = -\frac{1}{3}x+y-\frac{5}{3}} \\
f_2(1,2) = -\frac{1}{3}+2-\frac{5}{3} = 0 \rightarrow \boxed{f_2 \; is \; active}
\end{multline*} \\

\begin{multline*}
\underline{f_3(x,y) = x-4} \\
f_3(1,2) = 1-4<0 \rightarrow \boxed{f_3 \; is \; not \; active}
\end{multline*} \\

\begin{multline*}
\underline{f_4(x,y) = y-3} \\
f_4(1,2) = 2-3<0 \rightarrow \boxed{f_4 \; is \; not \; active}
\end{multline*} \\

\begin{multline*}
\underline{f_5(x,y) = -x} \\
f_5(1,2) = -1<0 \rightarrow \boxed{f_5 \; is \; not \; active}
\end{multline*} \\

\begin{multline*}
\underline{f_6(x,y) = -y} \\
f_6(1,2) = -2<0 \rightarrow \boxed{f_6 \; is \; not \; active}
\end{multline*} \\

\textbf{2.4} \\
The 2 active constraints are $f_1, f_2$. I've used the following link to "slide" the value of $u$:
\begin{lstlisting}[breaklines]
https://www.desmos.com/calculator/yf5btxc5bz
\end{lstlisting} 

Using it saw that as $u$ decreases in value, the "next" constraint to become active (while $f_1$ becomes inactive) is $f_5$. Hence we can calculate $u_{min}$: \\

$f_5(x,y)=0 \rightarrow -x=0 \rightarrow x=0$\\

$f_1(x,y)=0 \rightarrow -x+y-1=0 \rightarrow y=1$\\

$f_2(x,y)=u_{min} \rightarrow -\frac{1}{3}x+y-\frac{5}{3}=u_{min} =
-\frac{2}{3}$\\


Similarly, as $u$ increases in value, the constraint $f_4$ becomes active

$f_4(x,y)=0 \rightarrow y-3=0 \rightarrow y=3$\\

$f_1(x,y)=0 \rightarrow -x+y-1=0 \rightarrow x=2$\\

$f_2(x,y)=u_{max} \rightarrow -\frac{1}{3}x+y-\frac{5}{3}=u_{max} =
-\frac{2}{3}+3-\frac{5}{3} = \frac{2}{3}$\\


$u_{min} = -\frac{2}{3}, \; u_{max} = \frac{2}{3}$\\


\textbf{2.5} \\
we can calculate $x^*(u_{min}), x^*(u_{max})$ (for example by using the wolfram alpha link aforementioned and tweaking constraint $f_2$).\\
We get: \\
$x^*(u_{min}) = (0,1), x^*(u_{max}) = (2,3)$. In between we get that the minimum point "slides" on the graph of $f_1$, hence: \\

$x^*(u) = (x,x+1) \; (u \in [u_{min}, u_{max}], x \in [0,2])$\\

And the optimal value $p^*(u)$ is simply the value of the objective function ($f_0(x,y) = 0.5x-y$) at $x^*(u)$: \\

$p^*(u) = f_0(x^*(u)) \; (u \in [u_{min}, u_{max}], x \in [0,2])$\\

Of course that's not enough since we need to express $x^*(u), p^*(u)$ in terms of $u$. For that we'll remember that for $u \in [u_{min}, u_{max}]$, the active constraints are $f_1, f_2$ so we get:\\

$-x + y - 1 = 0 \rightarrow y=x+1$\\
$-\frac{1}{3}x + y - \frac{5}{3} - u = 0 \rightarrow
-\frac{1}{3}x + x + 1 - \frac{5}{3} - u = 0 \rightarrow
\frac{2}{3}x = u+\frac{2}{3} \rightarrow \\
x = \frac{3}{2}u+1, y = \frac{3}{2}u+2, \; 
\boxed{x^* = \begin{pmatrix}
           \frac{3}{2}u+1 \\
           \frac{3}{2}u+2 
         \end{pmatrix}}$\\
        

$\boxed{p^*(u) = \frac{1}{2} \left( \frac{3}{2}u+1 \right) - \frac{3}{2}u-2 = 
-\frac{3}{4}u-\frac{3}{2}}$\\

$u \in [-\frac{2}{3}, \frac{2}{3}]$ \\


\textbf{2.6} \\
$p^*(u)$ is differentiable at u=0. The derivative is:\\
$\frac{\partial p^*(u)}{\partial u} (0) = -\frac{3}{4}$\\
According to definition of optimal dual variable (lecture 6, slide 48): \\
$\lambda ^* = -\frac{\partial p^*(u)}{\partial u}(0) = \frac{3}{4}$


\textbf{2.7} \\
See 2 figures below. \\

\begin{figure}[h!]
\includegraphics[width=0.8\linewidth]{q2_7_u_min.PNG}
\caption{Q2.7 for $u_{min}=-\frac{2}{3}$: minimum point (0,1) is marked in blue}
\end{figure}

\begin{figure}[h!]
\includegraphics[width=0.8\linewidth]{q2_7_u_max.PNG}
\caption{Q2.7 for $u_{max}=\frac{2}{3}$: maximum point (2,3) is marked in blue}
\end{figure}

\textbf{2.8} \\



\end{document}\\